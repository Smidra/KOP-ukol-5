\input opmac
\chyph

\tit Seznámení se se zvolenou pokročilou iterativní metodou na problému batohu
\centerline{\bf 13. prosince 2020, Radek Šmíd}
\medskip
\centerline{\it https://github.com/Smidra/KOP-ukol-4}


% Specifikace úlohy
\sec Specifikace úlohy
Cílem úlohy bylo implementovat jeden z pokročilých algoritmů pro řešení problému batohu.\break
Na výběr bylo z:

\begitems
* Simulované ochlazování
* Genetické algoritmy
* Tabu search
\enditems

Po úspěšné implementaci algoritmu bylo za úkol provádět experiementy a seznámit se s pricipem nastavování parametrů. Proces experimentace jsem se pokusil zachytit v tomto dokumentu. Výsledné naměřené hodnoty pak prezentovat a pokusit se o interpretaci\fnote{https://moodle-vyuka.cvut.cz/mod/assign/view.php?id=89702}.


% Co je to simulované ochlazování
\sec Stručný popis zvoleného algoritmu
Zvolil jsem si algoritmus simulovaného ochlazování. Je to pravděpodobnostní metoda řešení problému založená na analogii se simulací ochlazování tuhnoucího kovu. Metoda prohledává stavový prostor a má tendenci zůstávat v lokálních minimech. Algoritmus nezaručuje nalezení globálního minima, ale správým nastavením jeho parametrů (teplota, chladící koeficient) můžeme docílit velmi přesných výsledků.


% Rozbor možných řešení + Jak jsem řešil já
\sec Rozbor řešení
Řešení znovu staví na základech položených v předchozích cvičeních. Instance problému batohu jsou drženy v paměti jako objekty třídy KnapsackInstance. Pro tuto úlohu jsem připsal ještě třídu {\bf KnapsackState}. Ta reprezentuje jeden stav nějaké instance problému batohu. Má metody:
\begitems
* is\_solution -- Vrací bool podle toho jestli je stav řešením problému.
* flip( int n ) -- Věc na pozici $n$ buď vyndá z batohu, nebo ji tam přidá.
* randomize -- Vyrobí náhodný, validní stav.
* random\_solution -- Vyrobí náhodné řešení.
* is\_better( state challenger ) -- Porovná, jestli je lepší tento stav nebo challenger.
* random\_neighbour\_solution -- Vrátí náhodný sousední stav, který je také řešení problému.
\enditems

\secc Stavový prostor
Jako stavový prostor problému batohu jsem zvolil všechna řešení problému. Algoritmus odmítá relaxovat vstoupením do nevalidního stavu. Pro získání co nejlepších vlastností podle přednášky je prospěšné začínat z náhodného stavu. Proto je třeba vygenerovat na začátek náhodné řešení.

\secc random\_solution
Tato funkce se použije při hledání startovacího stavu. V cyklu randomizuje stav a ptá se jestli je to validní řešení. Tento cyklus opakuje až do pětinásobku počtu věcí v batohu. Pokud se jí ani pak nepodaří najít validní řešení svojí práci vzdá a začne od triviálního řešení.

\secc Operace
Zvolené operace jsou změna jednoho bitu. Tedy buď přidání nebo odebrání věci z/do batohu takovým způsobem, že se neprotrhne. Tato operace je implementována metodou flip.

\secc random\_neighbour\_solution
Se použije ve funkci {\bf try}, kde vybíráme náhodného souseda. Tedy takový stav, pro který stačí použít jednou metodu flip, abychom ho dosáhli. Nalezený soused, ale samozřejmě musí být validní řešení. Prostor bude jistě spojitý, protože pokud z řešení odebereme věc, je to stále řešení. Postupným odebíráním věcí se dostaneme až do triviálního řešení, které máme k dispozici vždy. 

\secc Normalizace ceny
Aby nehrály roli jednotky ve kterých je cena uvedena (koruny/haléře), tak jsem rozdíl cen implementoval speciální funkcí. Ta vezme cenu prvního stavu ($c1$) a vypočítá konstantu $a$ pomocí které lze naškálovat $c1$ na 100. Cenu druhého stavu pak vynásobí stejnou konstantou. Od $100$ pak odečte $c2$ a získá normalizovaný rozdíl ceny stavů, který se nebude s jednotkami měnit. Tím získáme stabilnější algoritmus, protože abolutní velikosti cen nebudou mít vliv na teplotu při vyhodnocování funkce try.


\secc Ukončení
Funkce frozen vrátí True po třiceti kolech beze změny. Funkce equilibrium vrátí True po dvojnásobku počtu věcí v instanci.


\vfill\break
\sec Experimentace
Používám instance o velikosti 32. Prvotní nastavení parametrů bylo náhodné $t = 10$ a $c = 0.995$ (chladící koeficient) podle simulátoru. Sleduji vývoj ceny v grafu. Algoritmus došel k řešení 32944, což je správné řešení první instance.

% Obrázek Graf02.pdf
\centerline{ \picwidth=15cm \inspic{graf02.pdf} }
\caption/f Vývoj ceny stavu pro první experiment.
\bigskip

Podle grafu vidíme, že sice nejlepší stav algoritmus úspěšně našel, ale úplně nakonec se v něm neustálil. Velmi pozitivně vnímám, že jsme se dostali ke správnému řešení a k jasnému vzestupnému trendu grafu. Naopak negativní je pomalá konvergence k řešení a možná až příliš velká diverzifikace. Podíváme se na graf vývoje nejlepší ceny a podle něho se budeme dále rozhodovat. Program znovu spustíme, takže konkrétní hodnoty budou jiné. 

% Obrázek graf03.pdf
\centerline{ \picwidth=14cm \inspic{graf03.pdf} }
\caption/f Vývoj nejlepší ceny pro první experiment.
\bigskip

Pro druhý běh algoritmus skončil s nepatrnou chybou, těsně vedle správného řešení. Přišel na něj ale podle grafu poměrně brzy. Ze zjištěných hodnot uvažuji co bych mohl zlepšit. Snížením chladícího koeficientu bychom měli dosáhnout menší intezifikace a tedy možná méně \uv{šílené} křivky. To by v kombinaci s poměrně rychle nalezeným řešením mohlo dávat dobré výsledky. Nasavíme tedy chladící koeficient na $c=0.99$.

% Obrázek graf04.pdf
\centerline{ \picwidth=15cm \inspic{graf04.pdf} }
\caption/f Vývoj ceny stavu pro druhý experiment.
\bigskip

Algoritmus opět skončil s cenou 32941, tedy těsně vedle výsledku. Pravděpodobně se bude jednat o nešťastné lokální minimum. Zlepšením je dosažení stejného výsledku během přibližně polovičního počtu kroků. Stále to ale vypadá, že bychom mohli zmenšit diverzifikaci při pohledu na velké skoky cen stavů. Pokusíme se to udělat zmenšením startovací teploty na polovinu, tedy $t=5$.

% Obrázek graf05.pdf
\centerline{ \picwidth=14cm \inspic{graf05.pdf} }
\caption/f Vývoj ceny stavu pro třetí experiment.
\bigskip

Tento běh dospěl ke správné nejlepší ceně. Zase se nám zmenšil poče kroků. V grafu ale nevidíme tak sebevědomý růst křivky vzhůru, jako v předhozím příkladech. I přesto zkusíme jaké by to bylo cenu ještě jednou zmenšit. Tentokrát na 2. 
\vfill\break

% Obrázek graf06.pdf
\centerline{ \picwidth=15cm \inspic{graf06.pdf} }
\caption/f Vývoj ceny stavu pro čtvrtý experiment.
\bigskip

Řešení stále dává správné. Ale rozhodně bude třeba zvednout intenzifikaci. Heuristika se dostane do správného stavu, klidně i několikrát, ale nemá žádný problém z něj utéct. Tentokrát naopak zvedneme ochlazovací koeficient. $c=0.999$

% Obrázek graf07.pdf
\centerline{ \picwidth=15cm \inspic{graf07.pdf} }
\caption/f Vývoj ceny stavu pro pátý experiment.
\bigskip

Stále se držíme správných výsledků, což indikuje, že heuristika stále funguje a naše parametry nejsou příliš špatné. Bohužel podle grafu vypadá toto řešení opravdu špatně. Heuristika uteče v podstatě z každého minima a není vůbec vidět sebevědomá konvergence ke správnému řešení. Jako poslední experiment se ještě pokusíme podruhé snížit koeficient chlazení. Tentorktát na $c=0.9$.

% Obrázek graf08.pdf
\centerline{ \picwidth=17cm \inspic{graf08.pdf} }
\caption/f Vývoj ceny stavu pro šestý experiment.
\bigskip

Tento experiment trval zásadně nejkratší dobu kvůli velmi nízkému počtu kroků. Bohužel také dosál nejhoršího řešení (32723) pořád je to ale velmi blízko pravdě. Konečně jsme dosáhli menšího \uv{vyskakování} z minim. Také vidíme jak se hodnota na konci ustaluje. Celkově vzato toto vypadá zatím jako nejlepší nastavení, také proto, že je velmi rychlé. Provedeme několik běhů programu za sebou a prozkoumáme, jestli řešení nedává příliš často velké chyby.

\bigskip
\centerline{\table{r|l}{
Běh&Chyba\crl
1.&221\cr
2.&305\cr
3.&0\cr
4.&898\cr
5.&0\cr
}}\bigskip

Bohužel ukazuje se, že toto nastavení může dosahovat poměrně velkých chyb. Výsledné nastavení tedy přiblížíme již naměřenému $t=2$, $c=0.99$. Jako poslední nastavení tedy zvolíme $t=2$ a $c=0.97$.


% Obrázek graf09.pdf
\centerline{ \picwidth=17cm \inspic{graf09.pdf} }
\caption/f Vývoj ceny stavu pro sedmý experiment.
\bigskip

A ještě zkontolujeme namátkově jestli nedělá veliké chyby.

\bigskip
\centerline{\table{r|l}{
Běh&Chyba\crl
1.&0\cr
2.&0\cr
3.&78\cr
4.&0\cr
5.&108\cr
}}\bigskip

Nevypadá to, že by dělalo, uvidíme jak si toto nastavení povede s chybovostí při black box textování.


\vfill\break
% Přehledná prezentace výsledků
\sec Zhodnocení experimentů
% Graf čas + chybovost
% Graf postupného vývoje ceny


Zkusíme běh na 100 instancích o velikosti 32. Hodnoty srovnáme s dodaným referenčním řešením.

\bigskip
\centerline{\table{r|r|r}{
{\bf 2/0.97}&MAX&AVG\crl
Chyba&70.9\%&7.9\%\cr
Čas&2.91s&1.32s\cr
}}\bigskip


Ukazuje se, že čas je sice skvělý a průměrná chybovost také. Bohužel je ale maximální chyba opravdu příliš vysoko. Vraťme se tedy zpět k ladění parametrů. Pro zvýšení přesnosti na úkor času zvedneme původní teplotu na $t=5$. Celkové hodnoty po testování na 100 instancí se změnily na:

\bigskip
\centerline{\table{r|r|r}{
{\bf 5/0.97}&MAX&AVG\crl
Chyba& 51.4\%& 2.7\%\cr
Čas  & 8.7s& 3.0s\cr
}}\bigskip

To je bohužel stále moc veliká chybovost. Zkusme se podívat i na jiné parametry. Je možné, že mám příliš rychlou ukončovancí podmínku. Tedy, že se tavenina příliš rychle prohlásí za zmraženou. Místo třiceti kol bez žádné změny toto číslo rapidně zvednu na 300. Abych dosahoval výrazně lepší chybovosti razantně také zvednu teplotu na $t=30$.

% Obrázek graf09.pdf
\centerline{ \picwidth=17cm \inspic{graf19.pdf} }
\caption/f Vývoj ceny stavu pro osmý experiment.
\bigskip

Vypadá to, že jsme nalezli chybu. Heuristika se úspěšně usadila v nejlepším stavu. Přitom kroků bylo ještě přijatelné množství. Přejdeme tedy s touto nejnovější iterací parametrů zpátky do black--box testingu. Pustím heuristiku zase na 100 instancí a zaznamenám vlastnosti řešení.

\bigskip
\centerline{\table{r|r|r}{
{\bf 30/0.97}&MAX&AVG\crl
Chyba& 5.5\%& 0.6\%\cr
Čas  &  21.8s& 9.6s\cr
}}\bigskip

\hfill $\epsilon\upsilon\rho\eta\kappa\alpha$!

Řešení je nyní velmi uspokojivé, přijmeme ho a experimentaci ukončíme.

% Interpretace výsledků
\sec Závěr
V domácím úkolu jsem implementoval simulované ochlazování a pokusil se metodickým přístupem nalézt nejlepší parametry pro zadané NK--32 instance. Herustika velmi citlivě ragovala na změnu všech parametrů. Nejenom ale na dva hlavní (počáteční teplotu a ochlazovací koeficient). Je také zásadní mít korektně nastavené hodnoty ukončování pro metodu frozen. Zyšování počáteční teploty a chladícího koeficientu pomalu vylaďujeme intezifikaci a diverzifikaci heuristiky pro náš algoritmus. Vyzkoušel jsem si práci s pokročilou heuristikou a upravil svůj program na automatizaci úkonů spojených s vizualizací od pythonu až po matlab grafy. 




\bye
